\chapter{NEEM-Acquisition}
\label{ch:acquisition}
\chapterauthor{S. Koralewski, A. Hawkin}

This chapter focuses on the acquisition process of \neems.
At first, we will provide the tools and procedures to acquire episodic memories from robots performing experiments.
The second section focuses on the \neem acquisition from virtual reality. 

%Each section will contain an example \neem to provide insights on, how the representation, described in the chapters \ref{ch:background}, \ref{ch:narrative} and \ref{ch:experience} is utilized to capture performed activities by robots or by humans. 
%In addition, each example \neem is available on the \neemhub for downloading.



\section{Data Structure}

We are using MongoDB to capture the data structures of the \neems.
If you will use the \knowrob interface to create your \neems then your \neem will consist of at least 3 folders - \textit{annotations}, \textit{inferred} and \textit{triples}.
The \neemnar and \neemexp are stored as a collection of BSON \footnote{http://bsonspec.org/faq.html} files.
Each folder should contain a BSON file and metafile stored as JSON. The metafile will include additional information related to \neems. This additional meta information is useful for searching \neem on \openease platform and hence needs to be provided by \neem creator while \neem acquisition time. An example of such information is as displayed below:

\def\arraystretch{1.1}%
\begin{figure}[htb]
\begin{lstlisting}[language=json,firstnumber=1]
{
	"_id" : ObjectId("5f22b1f512db5aed7cd1961b"), 
	"created_by" : "seba",
	"created_at" : "2020-07-21T06:54:25+00:00",
	"model_version" : "0.1",
	"description" : "NEEM for robot making pizza.",
	"keywords" : [	
	"Pizza",
	"Robot"
	],
	"url" : "Placeholder for the NEEM hub repository url",
	"name" : "NEEM for robot making pizza",
	"activity" : {
		"name" : "Pizza making",
		"url" : "Placeholder for the url/uri of Activity concept defined in ontology"    
	},
	"environment" : "Kitchen",
	"image" : "placeholder for image url for showing neem image on openEASE",  
	"agent" : "Robot"
}
	
\end{lstlisting}
\caption{The meta data structure.}
\label{fig:meta_data}
\end{figure}
Each generated \neem stores also the complete state of the \soma ontology which was used during the acquisition process.
The benefit of this is that while loading a \neem, it is not required to keep track to load the correct \soma version.
In the following, we will give an overview which information is contained in those folders generated by \knowrob:


\begin{description}
	\item[\textbf{annotations}] The annotations collection contains annotations(comments) which are asserted to the concepts of the ontology.
	\item[\textbf{inferred}] The inferred collection contains triples which were inferred and not asserted during the logging process. Inference processes can be triggered when triples are asserted directly to the knowledge base.
	\item[\textbf{triples}] The triples collection contains all triples which were asserted into the knowledge base during run time.
\end{description}

\subsection{Triple data as JSON object}
	Triple data can also be provided in form of JSON documents, where triples are represented as subject, predicate and object. Subjects and objects are identified by an Internationalized Resource Identifier (IRI), which is pointing to concepts or instances defined in the \soma ontology. A triple can either link subject with an object, or can link subject with data value which is represented using one of the base types: string, boolean, and a number. Whereas, the predicate is named by the IRI pointing to property concepts mentioned in the \soma ontology. It is also possible to provide additional time scope fields `since` and `until` to indicate that the given triple is valid for the given time scope. These values are considered here in seconds from when an experiment has started being recorded. An example of such a JSON document is given below where the \emph{Salad\_PMRVYPJH} has a \emph{patient} role from 27.739th second till 29.075. By default triple is valid for the infinite time when the scope parameters are not specified. 
\def\arraystretch{1.1}%
\begin{figure}[htb]
\begin{lstlisting}[language=json,firstnumber=1]
[
 {
   "s": "http://www.ease-crc.org/ont/SOMA.owl#Salad_PMRVYPJH",
   "p": "http://www.ontologydesignpatterns.org/ont/dul/DUL.owl#hasRole",
   "o": "http://www.ease-crc.org/ont/SOMA.owl#Patient_YGJUVNDR",
   "since": 27.739,
   "until": 29.075
 }
]
\end{lstlisting}
\caption{The triple data structure.}
\label{fig:triple_data}
\end{figure}

It is important to note that, this is an intermediate data format which is not equivalent with how the NEEM narrative is actually stored in databases.
The format described here rather serves as an easy-to-use interchange format.

\input{content/acquisition/robot-neem/robot-neem}
\input{content/acquisition/vr-neem/vr-neem}
